\documentclass[12pt]{article}
\usepackage{amsmath}
\usepackage[utf8]{inputenc}
\begin{document}

\section*{Ejercicio 5}
Resolver la ecuación:
\[
5x + 9 - 3x - 7 + x = -3x + 2 + 6x
\]

\subsection*{Paso 1}
Juntamos todo lo que lleva "x" en un lado del signo igual y los números sueltos en el otro:
\[
5x - 3x + x + 3x - 6x = -9 + 7 + 2
\]

\subsection*{Paso 2}
Operamos ambos lados:
\[
0x = 0
\]
\[
0 = 0
\]

\subsection*{Paso 3}
Cuando pasa esto significa que la ecuación tiene infinitas soluciones.
\[
\text{infinitas soluciones}
\]
\end{document}