\documentclass[12pt]{article}
\usepackage{amsmath}
\usepackage[utf8]{inputenc}
\begin{document}

\section*{Ejercicio 2.a}
Resolver la ecuación:
\[
2x - 3 + 7x + 1 = -4x + 5 + 6x
\]

\subsection*{Paso 1}
Juntamos todo lo que lleva "x" en un lado del signo igual y al otro los números sueltos:
\[
2x + 7x + 4x - 6x = 5 + 3 - 1
\]

\subsection*{Paso 2}
Operamos ambos lados:
\[
7x = 7
\]

\subsection*{Paso 3}
Aislamos la incógnita:
\[
x = \frac{7}{7}
\]

\[
x = 1
\]
\end{document}