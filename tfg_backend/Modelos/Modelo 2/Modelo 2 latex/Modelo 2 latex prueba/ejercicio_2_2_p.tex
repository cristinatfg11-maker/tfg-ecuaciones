\documentclass[12pt]{article}
\usepackage{amsmath}
\usepackage[utf8]{inputenc}
\begin{document}

\section*{Ejercicio}
Resolver la ecuación:
\[
10x + 18 - 6x - 14 + 2x = -6x + 4 + 12x
\]

\subsection*{Paso 1}
Juntamos todo lo que lleva "x" en un lado del igual y los números sueltos en el otro:
\[
10x - 6x + 2x + 6x - 12x = 4 - 18 + 14
\]

\subsection*{Paso 2}
Operamos ambos lados:
\[
0x = 0
\]
\[
0 = 0
\]

\subsection*{Solución}
Se trata de una identidad:
\[
\text{infinitas soluciones}
\]
\end{document}