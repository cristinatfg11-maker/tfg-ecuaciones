\documentclass[12pt]{article}
\usepackage{amsmath}
\usepackage[utf8]{inputenc}
\begin{document}

\section*{Ejercicio 2.4}
Resolver la ecuación:
\[
6x - 10 - x - 2x + 3 = -7 - 3x + 5
\]

\subsection*{Paso 1}
Juntamos todo lo que lleva "x" en un lado del igual y los números sueltos en el otro:
\[
6x - x - 2x + 3x = -7 + 5 + 10 - 3
\]

\subsection*{Paso 2}
Operamos ambos lados:
\[
6x = 5
\]

\subsection*{Solución}
Aislamos la incógnita:
\[
x = \frac{5}{6}
\]

\[
x = \frac{5}{6}
\]
\end{document}