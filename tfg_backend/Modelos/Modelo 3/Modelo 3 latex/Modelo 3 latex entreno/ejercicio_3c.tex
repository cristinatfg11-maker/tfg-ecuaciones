\documentclass[12pt]{article}
\usepackage{amsmath}
\usepackage[utf8]{inputenc}
\begin{document}

\section*{Ejercicio 3c}
Resolver la ecuación:
\[
4x - (x - 7) + 6(2x - 1) = 3(5x + 2)
\]

\subsection*{Paso 1}
Primero eliminamos los paréntesis, teniendo en cuenta los signos:
\[
4x - x + 7 + 12x - 6 = 15x + 6
\]

\subsection*{Paso 2}
Juntamos todos los términos con "x" a un lado y los números al otro:
\[
4x - x + 12x - 15x = 6 - 7 + 6
\]

\subsection*{Paso 3}
Operamos:
\[
0x = 5
\]
\[
0 = 5
\]

\subsection*{Paso 4}
Caso especial: cuando el resultado es una igualdad falsa, la ecuación no tiene solución.
\[
\text{no tiene solución}
\]
\end{document}