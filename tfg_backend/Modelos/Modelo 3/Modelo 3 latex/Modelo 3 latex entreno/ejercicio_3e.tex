\documentclass[12pt]{article}
\usepackage{amsmath}
\usepackage[utf8]{inputenc}
\begin{document}

\section*{Ejercicio 3e}
Resolver la ecuación:
\[
4[6x - (x - 3) - (5 - 2x)] = 3(7x - 1) + 8x
\]

\subsection*{Paso 1}
Primero eliminamos los paréntesis ( ) y luego los corchetes [ ].
\[
4[6x - x + 3 - 5 + 2x] = 21x - 3 + 8x
\]
\[
4[7x - 2] = 21x - 3 + 8x
\]
\[
28x - 8 = 29x - 3
\]

\subsection*{Paso 2}
Juntamos los términos que llevan "x" en un lado de la igualdad y los números sueltos al otro:
\[
28x - 29x = -3 + 8
\]

\subsection*{Paso 3}
Operamos:
\[
-x = 5
\]

\subsection*{Paso 4}
Aislamos la incógnita y simplificamos:
\[
x = \frac{5}{-1}
\]

\[
x = -5
\]
\end{document}