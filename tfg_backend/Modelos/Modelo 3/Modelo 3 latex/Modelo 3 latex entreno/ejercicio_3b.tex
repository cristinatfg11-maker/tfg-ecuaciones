\documentclass[12pt]{article}
\usepackage{amsmath}
\usepackage[utf8]{inputenc}
\begin{document}

\section*{Ejercicio 3b}
Resolver la ecuación:
\[
-2 + 4(2x + 5) = -2x + 18 - 6x
\]

\subsection*{Paso 1}
Primero eliminamos los paréntesis, teniendo en cuenta los signos:
\[
-2 + 8x + 20 = -2x + 18 - 6x
\]

\subsection*{Paso 2}
Juntamos todos los términos con "x" a un lado y los números al otro:
\[
8x + 2x + 6x = 18 + 2 - 20
\]

\subsection*{Paso 3}
Operamos a ambos lados:
\[
16x = 0
\]

\subsection*{Paso 4}
Aislamos la incógnita:
\[
x = \frac{0}{16}
\]

\[
x = 0
\]
\end{document}