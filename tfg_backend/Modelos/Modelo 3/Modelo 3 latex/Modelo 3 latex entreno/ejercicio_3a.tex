\documentclass[12pt]{article}
\usepackage{amsmath}
\usepackage[utf8]{inputenc}
\begin{document}

\section*{Ejercicio 3a}
Resolver la ecuación:
\[
2(x - 3) + 5(-4 + 3x) = -6(x + 3)
\]

\subsection*{Paso 1}
Eliminamos los paréntesis, teniendo en cuenta que lo que tenemos delante de cada paréntesis afecta a todo su contenido:
\[
2x - 6 - 20 + 15x = -6x - 18
\]

\subsection*{Paso 2}
Juntamos todos los términos con "x" a un lado, y los números sueltos al otro:
\[
2x + 15x + 6x = -18 + 6 + 20
\]

\subsection*{Paso 3}
Operamos a ambos lados:
\[
23x = 8
\]

\subsection*{Paso 4}
Aislamos la incógnita:
\[
x = \frac{8}{23}
\]

\[
x = \frac{8}{23}
\]
\end{document}