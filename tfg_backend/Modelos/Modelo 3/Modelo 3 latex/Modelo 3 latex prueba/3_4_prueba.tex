\documentclass[12pt]{article}
\usepackage{amsmath}
\usepackage[utf8]{inputenc}
\begin{document}

\section*{Ejercicio 3.4 p}
Ecuación:
\[
1 - (9x + 3) + 7x = -5(x + 2) + 3x
\]

\subsection*{Paso 1}
Primero eliminamos los paréntesis, teniendo en cuenta los signos:
\[
1 - 9x - 3 + 7x = -5x - 10 + 3x
\]

\subsection*{Paso 2}
Juntamos las "x" en un lado del signo igual y los números sueltos en el otro:
\[
-9x + 7x + 5x - 3x = -10 + 3 - 1
\]

\subsection*{Paso 3}
Operamos:
\[
0x = -8
\]
\[
0 = -8
\]

\subsection*{Paso 4}
Como la igualdad es falsa, la ecuación no tiene solución.
\[
\text{no tiene solución}
\]
\end{document}