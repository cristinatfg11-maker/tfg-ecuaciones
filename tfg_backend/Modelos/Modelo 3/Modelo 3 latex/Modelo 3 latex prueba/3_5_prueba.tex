# Nombre de archivo: 3_5_prueba.tex
# Versión: FIX_STEPS_BREAKDOWN_V1.0

\documentclass[12pt]{article}
\usepackage{amsmath}
\usepackage[utf8]{inputenc}
\begin{document}

\section*{Ejercicio 3.5 p}
Ecuación:
\[
-6x + 5[(1 - 4x) + 9(-2 - 2x)] = 0
\]

\subsection*{Paso 1}
Primero eliminamos los paréntesis ( ), pero en este aparecen ( ) y [ ]; siempre empezamos por ( ) y luego [ ].

1.1) Eliminamos paréntesis curvos:
\[
-6x + 5[1 - 4x - 18 - 18x] = 0
\]

1.2) Eliminamos corchetes (multiplicando por 5):
\[
-6x + 5 - 20x - 90 - 90x = 0
\]

\subsection*{Paso 2}
Juntamos las "x" en un lado del signo igual y los números sueltos en el otro:
\[
-6x - 20x - 90x = -5 + 90
\]

\subsection*{Paso 3}
Operamos:
\[
-116x = 85
\]

\subsection*{Paso 4}
Aislamos la incógnita "x" y simplificamos:
\[
x = \frac{85}{-116}
\]

\[
x = -\frac{85}{116}
\]
\end{document}