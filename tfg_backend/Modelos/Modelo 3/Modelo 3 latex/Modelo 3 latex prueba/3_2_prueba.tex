\documentclass[12pt]{article}
\usepackage{amsmath}
\usepackage[utf8]{inputenc}
\begin{document}

\section*{Ejercicio 3.2 p}
Resolver la ecuación:
\[
1 + 4(6 - 2x) - (x - 3) = 5x
\]

\subsection*{Paso 1}
Primero eliminamos los paréntesis, teniendo en cuenta los signos:
\[
1 + 24 - 8x - x + 3 = 5x
\]

\subsection*{Paso 2}
Juntamos las "x" en un lado del igual y los números sueltos en el otro:
\[
-8x - x - 5x = -1 - 24 - 3
\]

\subsection*{Paso 3}
Operamos:
\[
-14x = -28
\]

\subsection*{Paso 4}
Aislamos la incógnita y simplificamos:
\[
x = \frac{-28}{-14}
\]

\[
x = 2
\]
\end{document}