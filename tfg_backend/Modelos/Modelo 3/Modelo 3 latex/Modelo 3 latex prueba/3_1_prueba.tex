\documentclass[12pt]{article}
\usepackage{amsmath}
\usepackage[utf8]{inputenc}
\begin{document}

\section*{Ejercicio 3.1 p}
Resolver la ecuación:
\[
6 - (5 - 2x) + 7(x - 4) = 3 - 6x
\]

\subsection*{Paso 1}
Primero eliminamos los paréntesis, teniendo en cuenta los signos:
\[
6 - 5 + 2x + 7x - 28 = 3 - 6x
\]

\subsection*{Paso 2}
Juntamos los términos que contienen "x" en un lado y los números sueltos en el otro:
\[
2x + 7x + 6x = 3 - 6 + 5 + 28
\]

\subsection*{Paso 3}
Operamos:
\[
15x = 30
\]

\subsection*{Paso 4}
Aislamos la incógnita y simplificamos:
\[
x = \frac{30}{15}
\]

\[
x = 2
\]
\end{document}