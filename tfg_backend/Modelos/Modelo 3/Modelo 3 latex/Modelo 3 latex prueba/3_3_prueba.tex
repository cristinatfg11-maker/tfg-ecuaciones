# Nombre de archivo: 3_3_prueba.tex
# Versión: FIX_STEPS_BREAKDOWN_V1.0

\documentclass[12pt]{article}
\usepackage{amsmath}
\usepackage[utf8]{inputenc}
\begin{document}

\section*{Ejercicio 3.3 p}
Ecuación:
\[
7 - [5x + 2(-4x + 8)] = 3(-10 - x)
\]

\subsection*{Paso 1}
Primero eliminamos los paréntesis ( ), pero en este aparecen ( ) y [ ]; siempre empezamos por ( ) y luego [ ].

1.1) Eliminamos paréntesis curvos:
\[
7 - [5x - 8x + 16] = -30 - 3x
\]

1.2) Eliminamos corchetes (el signo menos cambia los signos de dentro):
\[
7 - 5x + 8x - 16 = -30 - 3x
\]

\subsection*{Paso 2}
Juntamos las "x" en un lado del igual y los números sueltos en el otro:
\[
-5x + 8x + 3x = -30 - 7 + 16
\]

\subsection*{Paso 3}
Operamos:
\[
6x = -21
\]

\subsection*{Paso 4}
Aislamos la incógnita y simplificamos:
\[
x = \frac{-21}{6}
\]

\[
x = -\frac{7}{2}
\]
\end{document}