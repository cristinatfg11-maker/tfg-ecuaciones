\documentclass[12pt]{article}
\usepackage{amsmath}
\usepackage[utf8]{inputenc}
\begin{document}

\section*{Ejercicio 1.8}
Resolver la siguiente ecuación:
\[
 -m = -5
\]

\subsection*{Paso 1}
Es un caso especial: la incógnita debe quedar sola y positiva. Observamos que está multiplicada por -1, por tanto pasa al otro lado dividiendo, sin cambiar el signo:
\[
m = \frac{-5}{-1}
\]

\subsection*{Paso 2}
Operamos:
\[
m = 5
\]

\[
m = 5
\]
\end{document}