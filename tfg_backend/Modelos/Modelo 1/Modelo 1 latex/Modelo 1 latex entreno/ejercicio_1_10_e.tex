\documentclass[12pt]{article}
\usepackage{amsmath}
\usepackage[utf8]{inputenc}
\begin{document}

\section*{Ejercicio 1.10}
Resolver la siguiente ecuación:
\[
\frac{Z}{8} - 3 = 7
\]

\subsection*{Paso 1}
Aislamos la incógnita "Z". Primero, el número que está restando pasa al otro lado sumando:
\[
\frac{Z}{8} = 7 + 3
\]
\[
\frac{Z}{8} = 10
\]

\subsection*{Paso 2}
Segundo, el número que está dividiendo pasa al otro lado multiplicando:
\[
Z = 10 \times 8
\]

\subsection*{Paso 3}
Operamos:
\[
Z = 80
\]

\[
Z = 80
\]
\end{document}