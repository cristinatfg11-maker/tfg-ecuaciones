\documentclass[12pt]{article}
\usepackage{amsmath}
\usepackage[utf8]{inputenc}
\begin{document}

\section*{Ejercicio 1.10 p}
Resolver la siguiente ecuación:
\[
\frac{m}{2} - 9 = -9
\]

\subsection*{Paso 1}
Aislamos la incógnita "m". Primero, el número que está restando pasa sumando:
\[
\frac{m}{2} = -9 + 9
\]
\[
\frac{m}{2} = 0
\]

\subsection*{Paso 2}
Segundo, el número que está dividiendo pasa multiplicando:
\[
m = 0 \times 2
\]

\subsection*{Paso 3}
Operamos:
\[
m = 0
\]

\[
m = 0
\]
\end{document}