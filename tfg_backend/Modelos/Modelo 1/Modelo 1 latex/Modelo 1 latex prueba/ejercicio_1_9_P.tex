\documentclass[12pt]{article}
\usepackage{amsmath}
\usepackage[utf8]{inputenc}
\begin{document}

\section*{Ejercicio 1.9 p}
Resolver la siguiente ecuación:
\[
\frac{X}{9} - 3 = 7
\]

\subsection*{Paso 1}
Aislamos la incógnita "X". Primero, el número que está restando pasa sumando:
\[
\frac{X}{9} = 7 + 3
\]
\[
\frac{X}{9} = 10
\]

\subsection*{Paso 2}
Segundo, el número que está dividiendo pasa multiplicando:
\[
X = 10 \times 9
\]

\subsection*{Paso 3}
Operamos:
\[
X = 90
\]

\[
X = 90
\]
\end{document}