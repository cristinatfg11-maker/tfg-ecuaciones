\documentclass[12pt]{article}
\usepackage{amsmath}
\usepackage{amssymb}
\usepackage[utf8]{inputenc}
\usepackage[spanish]{babel}

\begin{document}

\section*{Ejercicio 5.5}
Resolver la ecuación:
\[
-\frac{5x}{6} - \frac{3(x+3)}{4} = x + \frac{5(2x+3)}{15}
\]

\subsection*{Paso 1}
Buscamos el m.c.m. de los denominadores:
\[
\text{m.c.m.} = 60
\]

\subsection*{Paso 2}
Expresamos todas las fracciones con denominador común:
\[
\frac{-50x}{60} - \frac{45(x+3)}{60} = \frac{60x}{60} + \frac{20(2x+3)}{60}
\]

\subsection*{Paso 3}
Eliminamos denominadores, agrupamos las x en un lado del = y los núm. sueltos en el otro.
\[
-50x - 45x - 135 = 60x + 40x + 60
\]
\[
-50x - 45x - 60x - 40x = 60 + 135
\]

\subsection*{Paso 4}
Operamos, aislamos la incógnita y simplificamos:
\[
-195x = 195
\]
\[
x = \frac{195}{-195}
\]

\[
x = -1
\]
\end{document}