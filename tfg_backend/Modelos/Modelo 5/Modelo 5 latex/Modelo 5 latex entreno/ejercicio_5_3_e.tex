\documentclass[12pt]{article}
\usepackage[spanish]{babel}
\usepackage{amsmath}
\usepackage{amssymb}

\begin{document}

\section*{Ejercicio 5.3}
Resolver la ecuación:
\[
2x + 80x + 4x - 20x - 25x = -60 + 30 + 37 + 40 + 2
\]

\subsection*{Paso 1}
Buscamos el m.c.m. de los denominadores:
\[
\text{No aplica (no hay denominadores)}
\]

\subsection*{Paso 2}
Expresamos todas las fracciones con denominador común:
\[
\text{No aplica}
\]

\subsection*{Paso 3}
Eliminamos denominadores, agrupamos las x en un lado del = y los núm. sueltos en el otro.
\[
2x + 80x + 4x - 20x - 25x = -60 + 30 + 37 + 40 + 2
\]

\subsection*{Paso 4}
Operamos, aislamos la incógnita y simplificamos:
\[
41x = 49
\]
\[
x = \frac{49}{41}
\]

\[
x = \frac{49}{41}
\]
\end{document}