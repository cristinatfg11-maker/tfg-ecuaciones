\documentclass[12pt]{article}
\usepackage{amsmath}
\usepackage{amssymb}
\usepackage[utf8]{inputenc}

\begin{document}

\section*{Ejercicio 5.2}
Resolver la ecuación:
\[
\frac{5(1-2x)}{18} + \frac{7(x+2)}{12} = x + 3
\]

\subsection*{Paso 1}
Buscamos el m.c.m. de los denominadores:
\[
\text{m.c.m.} = 36
\]

\subsection*{Paso 2}
Expresamos todas las fracciones con denominador común:
\[
\frac{10(1-2x)}{36} + \frac{21(x+2)}{36} = \frac{36(x+3)}{36}
\]

\subsection*{Paso 3}
Eliminamos denominadores, agrupamos las x en un lado del = y los núm. sueltos en el otro.
\[
10 - 20x + 21x + 42 = 36x + 108
\]
\[
-20x + 21x - 36x = 108 - 10 - 42
\]

\subsection*{Paso 4}
Operamos, aislamos la incógnita y simplificamos:
\[
-35x = 56
\]
\[
x = \frac{56}{-35}
\]

\[
x = -\frac{8}{5}
\]
\end{document}