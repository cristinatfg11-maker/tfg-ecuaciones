\documentclass[12pt]{article}
\usepackage{amsmath}
\usepackage{amssymb}
\usepackage[utf8]{inputenc}
\usepackage[spanish]{babel}

\begin{document}

\section*{Ejercicio 5.1}
Resuelve la ecuación:
\[
\frac{3x}{8} + 2 = \frac{4}{5} - \frac{3(2x-7)}{10}
\]

\subsection*{Paso 1}
Buscamos el m.c.m. de los denominadores:
\[
\text{m.c.m.} = 40
\]

\subsection*{Paso 2}
Expresamos todas las fracciones con denominador común:
\[
\frac{15x}{40} + \frac{80}{40} = \frac{32}{40} - \frac{12(2x-7)}{40}
\]

\subsection*{Paso 3}
Eliminamos denominadores, agrupamos las x en un lado del = y los núm. sueltos en el otro.
\[
15x + 80 = 32 - 24x + 84
\]
\[
15x + 24x = 116 - 80
\]

\subsection*{Paso 4}
Operamos, aislamos la incógnita y simplificamos:
\[
39x = 36
\]
\[
x = \frac{36}{39}
\]

\[
x = \frac{12}{13}
\]
\end{document}