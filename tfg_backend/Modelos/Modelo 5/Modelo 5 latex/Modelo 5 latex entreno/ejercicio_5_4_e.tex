\documentclass[12pt]{article}
\usepackage{amsmath}
\usepackage{amssymb}
\usepackage[utf8]{inputenc}

\begin{document}

\section*{Ejercicio 5.4}
Resolvamos la ecuación:
\[
-3x + \frac{5(x-2)}{2} = \frac{4(10-5x)}{5} + 1
\]

\subsection*{Paso 1}
Buscamos el m.c.m. de los denominadores:
\[
\text{m.c.m.} = 10
\]

\subsection*{Paso 2}
Expresamos todas las fracciones con denominador común:
\[
\frac{-30x}{10} + \frac{25(x-2)}{10} = \frac{8(10-5x)}{10} + \frac{10}{10}
\]

\subsection*{Paso 3}
Eliminamos denominadores, agrupamos las x en un lado del = y los núm. sueltos en el otro.
\[
-30x + 25x - 50 = 80 - 40x + 10
\]
\[
-30x + 25x + 40x = 80 + 10 + 50
\]

\subsection*{Paso 4}
Operamos, aislamos la incógnita y simplificamos:
\[
35x = 140
\]
\[
x = \frac{140}{35}
\]

\[
x = 4
\]
\end{document}