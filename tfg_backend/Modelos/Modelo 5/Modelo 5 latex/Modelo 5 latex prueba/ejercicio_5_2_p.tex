\documentclass[12pt]{article}
\usepackage{amsmath}
\usepackage[utf8]{inputenc}

\begin{document}

\section*{Modelo 5.2}
Resolver la ecuación:
\[
\frac{3(x-2)}{5} + \frac{2(-3x+1)}{5} - \frac{2}{5} = \frac{-4x+3}{15} + \frac{16}{3}
\]

\subsection*{Paso 1}
Buscamos el m.c.m. de los denominadores:
\[
\text{m.c.m.} = 15
\]

\subsection*{Paso 2}
Expresamos todas las fracciones con denominador común:
\[
\frac{9(x-2)}{15} + \frac{6(-3x+1)}{15} - \frac{6}{15} = \frac{-4x+3}{15} + \frac{80}{15}
\]

\subsection*{Paso 3}
Eliminamos denominadores, agrupamos las x en un lado del = y los núm. sueltos en el otro.
\[
9x - 18 - 18x + 6 - 6 = -4x + 3 + 80
\]
\[
9x - 18x + 4x = 83 + 18
\]

\subsection*{Paso 4}
Operamos, aislamos la incógnita y simplificamos:
\[
-5x = 101
\]
\[
x = \frac{101}{-5}
\]

\[
x = -\frac{101}{5}
\]
\end{document}