\documentclass[12pt]{article}
\usepackage{amsmath}
\usepackage{amsfonts}
\usepackage[utf8]{inputenc}

\begin{document}

\section*{Modelo 5.5}
Resolver la ecuación:
\[
\frac{5 - x}{2} - 2 = \frac{1 - x}{2} - \frac{2(x+1)}{3}
\]

\subsection*{Paso 1}
Buscamos el m.c.m. de los denominadores:
\[
\text{m.c.m.} = 6
\]

\subsection*{Paso 2}
Expresamos todas las fracciones con denominador común:
\[
\frac{3(5-x)}{6} - \frac{12}{6} = \frac{3(1-x)}{6} - \frac{4(x+1)}{6}
\]

\subsection*{Paso 3}
Eliminamos denominadores, agrupamos las x en un lado del = y los núm. sueltos en el otro.
\[
15 - 3x - 12 = 3 - 3x - 4x - 4
\]
\[
-3x + 3x + 4x = 3 - 4 - 15 + 12
\]

\subsection*{Paso 4}
Operamos, aislamos la incógnita y simplificamos:
\[
4x = -4
\]
\[
x = \frac{-4}{4}
\]

\[
x = -1
\]
\end{document}