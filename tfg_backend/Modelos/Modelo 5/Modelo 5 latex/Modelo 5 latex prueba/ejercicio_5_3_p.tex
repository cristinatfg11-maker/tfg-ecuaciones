\documentclass[12pt]{article}
\usepackage{amsmath}
\usepackage[utf8]{inputenc}

\begin{document}

\section*{Ejercicio}
Resolver la ecuación:
\[
\frac{2(5x+2)}{9} - \frac{4x-1}{2} = x
\]

\subsection*{Paso 1}
Buscamos el m.c.m. de los denominadores:
\[
\text{m.c.m.} = 18
\]

\subsection*{Paso 2}
Expresamos todas las fracciones con denominador común:
\[
\frac{4(5x+2)}{18} - \frac{9(4x-1)}{18} = \frac{18x}{18}
\]

\subsection*{Paso 3}
Eliminamos denominadores, agrupamos las x en un lado del = y los núm. sueltos en el otro.
\[
20x + 8 - 36x + 9 = 18x
\]
\[
20x - 36x - 18x = -9 - 8
\]

\subsection*{Paso 4}
Operamos, aislamos la incógnita y simplificamos:
\[
-34x = -17
\]
\[
x = \frac{-17}{-34}
\]

\[
x = \frac{1}{2}
\]
\end{document}