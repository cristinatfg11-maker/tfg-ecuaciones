\documentclass[12pt]{article}
\usepackage{amsmath,amssymb}
\usepackage{lmodern}
\usepackage[T1]{fontenc}
\usepackage[utf8]{inputenc}
\usepackage{hyperref}

\begin{document}

\section*{Ejercicio}
Resolver la ecuación:
\[
\frac{8-4x}{3} - 2(5x+8) = \frac{2(4x+6)}{9} + 2(10x+1)
\]

\subsection*{Paso 1}
Buscamos el m.c.m. de los denominadores:
\[
\text{m.c.m.} = 9
\]

\subsection*{Paso 2}
Expresamos todas las fracciones con denominador común:
\[
\frac{3(8-4x)}{9} - \frac{18(5x+8)}{9} = \frac{2(4x+6)}{9} + \frac{18(10x+1)}{9}
\]

\subsection*{Paso 3}
Eliminamos denominadores, agrupamos las x en un lado del = y los núm. sueltos en el otro.
\[
24 - 12x - 90x - 144 = 8x + 12 + 180x + 18
\]
\[
-12x - 90x - 8x - 180x = 12 + 18 - 24 + 144
\]

\subsection*{Paso 4}
Operamos, aislamos la incógnita y simplificamos:
\[
-290x = 150
\]
\[
x = \frac{150}{-290}
\]

\[
x = -\frac{15}{29}
\]
\end{document}