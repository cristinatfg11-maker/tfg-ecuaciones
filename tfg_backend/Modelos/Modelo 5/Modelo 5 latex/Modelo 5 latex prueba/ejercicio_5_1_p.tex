\documentclass[12pt]{article}
\usepackage[utf8]{inputenc}
\usepackage[spanish]{babel}
\usepackage{amsmath}

\begin{document}

\section*{Modelo 5.1 — Ejercicio}
Resolver la ecuación:
\[
5 - \frac{2(1-x)}{3} = \frac{2(x-1)}{4}
\]

\subsection*{Paso 1}
Buscamos el m.c.m. de los denominadores:
\[
\text{m.c.m.} = 12
\]

\subsection*{Paso 2}
Expresamos todas las fracciones con denominador común:
\[
\frac{60}{12} - \frac{8(1-x)}{12} = \frac{6(x-1)}{12}
\]

\subsection*{Paso 3}
Eliminamos denominadores, agrupamos las x en un lado del = y los núm. sueltos en el otro.
\[
60 - 8 + 8x = 6x - 6
\]
\[
8x - 6x = -6 - 60 + 8
\]

\subsection*{Paso 4}
Operamos, aislamos la incógnita y simplificamos:
\[
2x = -58
\]
\[
x = \frac{-58}{2}
\]

\[
x = -29
\]
\end{document}