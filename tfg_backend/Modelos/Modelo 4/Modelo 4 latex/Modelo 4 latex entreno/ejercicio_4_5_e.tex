\documentclass[12pt]{article}
\usepackage{amsmath}
\usepackage[spanish]{babel}
\begin{document}

\section*{Ejercicio 4.5}
Resolver la ecuación:
\[
\frac{2x+1}{7} + \frac{x-2}{4} = \frac{3x-5}{14}
\]

\subsection*{Paso 1}
Búsqueda del m.c.m.:
\[
\text{m.c.m.} = 28
\]

\subsection*{Paso 2}
Fracciones equivalentes con denominador común:
\[
\frac{4(2x+1)}{28} + \frac{7(x-2)}{28} = \frac{2(3x-5)}{28}
\]

\subsection*{Paso 3}
Eliminamos denominadores y operamos:
\[
8x + 4 + 7x - 14 = 6x - 10
\]

\subsection*{Paso 4}
Agrupamos:
\[
8x + 7x - 6x = -10 - 4 + 14
\]
\[
9x = 0
\]

\subsection*{Paso 5}
Aislamos la incógnita:
\[
x = \frac{0}{9}
\]

\[
x = 0
\]
\end{document}