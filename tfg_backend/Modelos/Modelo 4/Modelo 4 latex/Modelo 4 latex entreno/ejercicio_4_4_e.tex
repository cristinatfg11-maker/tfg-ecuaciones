\documentclass[12pt]{article}
\usepackage{amsmath}
\usepackage[spanish]{babel}
\begin{document}

\section*{Ejercicio: Resolver la ecuación}
\[
\frac{x-2}{10} - \frac{x+3}{4} = \frac{2x-7}{8}
\]

\subsection*{Paso 1}
Encontrar el m.c.m. de los denominadores:
\[
\text{m.c.m.} = 40
\]

\subsection*{Paso 2}
Convertir todas las fracciones al mismo denominador:
\[
\frac{4(x-2)}{40} - \frac{10(x+3)}{40} = \frac{5(2x-7)}{40}
\]

\subsection*{Paso 3}
Eliminar denominadores y desarrollar:
\[
4x - 8 - 10x - 30 = 10x - 35
\]

\subsection*{Paso 4}
Agrupamos términos semejantes:
\[
4x - 10x - 10x = 8 + 30 - 35
\]
\[
-16x = 3
\]

\subsection*{Paso 5}
Aislamos la incógnita:
\[
x = \frac{3}{-16}
\]

\[
x = -\frac{3}{16}
\]
\end{document}