\documentclass[12pt]{article}
\usepackage{amsmath}
\usepackage[utf8]{inputenc}
\begin{document}

\section*{Ejercicio: Resolver la ecuación}
Resolver:
\[
\frac{3x-4}{2} = \frac{2x+4}{3}
\]

\subsection*{Paso 1}
Buscamos el m.c.m. de los denominadores:
\[
\text{m.c.m.} = 6
\]

\subsection*{Paso 2}
Encontramos fracciones equivalentes con denominador común:
\[
\frac{3(3x-4)}{6} = \frac{2(2x+4)}{6}
\]

\subsection*{Paso 3}
Eliminamos denominadores y operamos:
\[
9x - 12 = 4x + 8
\]
Agrupamos términos:
\[
9x - 4x = 8 + 12
\]

\subsection*{Paso 4}
Operamos:
\[
5x = 20
\]

\subsection*{Paso 5}
Aislamos la incógnita y simplificamos:
\[
x = \frac{20}{5}
\]

\[
x = 4
\]
\end{document}